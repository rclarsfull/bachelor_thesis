\chapter{Einleitung}
\label{ch:intro}
Diese Arbeit wurde in Rahmen des Dualen Studiums der Informatik an der Hochschule Darmstadt in Kooperation mit der \ac{DFS} erstellt.
Die Inhalte dieser Arbeit basieren zu Teil auf den Anforderungen, Erfahrungen und Erkenntnissen, die im Rahmen Praxisprojekts der dritten Praxisphase in dem Team 
TI/TD der \ac{DFS} gesammelt wurden.

%
% Section: Motivation
%
\section{Motivation}
\label{sec:intro:motivation}
Das Team TI/TD der \ac{DFS} entwickelt Simulationsumgebungen, die in der Ausbildung von Fluglotsen eingesetzt werden. 
Während der Simulatorübungen werden Trainee-Lotsen von Adjazenlotsen unterstützt, die eine Vielzahl von Aufgaben gleichzeitig bewältigen müssen. 
Dies führt zu einer hohen kognitiven Belastung.
\\

Adjazenlotsen steuern nicht nur den Simulator, sondern übernehmen auch Kommunikationsaufgaben und unterstützen bei der Überwachung des Flugverkehrs in angrenzenden Sektoren. 
Diese Aufgaben erfordern ständige Aufmerksamkeit und schnelle Reaktionen.
\\

Um die kognitive Belastung zu reduzieren und die Effizienz der Adjazenlotsen zu steigern, untersucht diese Arbeit die Konzeption und Evaluation eines \ac{RL}-basierten Systems, 
das die Adjazenlotsen bei ihren Aufgaben unterstützt.
%
% Section: Ziele
%
\section{Ziel der Arbeit}
\label{sec:intro:goal}
Ziel dieser Arbeit ist die Konzeption und Evaluation eines \ac{RL}-Systems, das die konfliktfreie Steuerung des Flugverkehrs in den umliegenden Sektoren der Trainees übernimmt. 
Das System soll Pilot Commands generieren, um Flugkonflikte zu vermeiden und den Verkehr effizient zu steuern. 
Dabei werden die Realisierbarkeit der Anweisungen, physikalische Limitierungen und die Anweisungsfrequenz berücksichtigt. 
Die Evaluation soll die Funktionalität und Effizienz des Systems überprüfen und mögliche Verbesserungspotenziale aufzeigen.

%
% Section: Struktur der Arbeit
%
\section{Gliederung}
\label{sec:intro:structure}
Zu Beginn der Arbeit wird in Kapitel \ref{ch:fs_grundlagen} die Grundlagen der Problemdomäne Flugverkehrssteuerung und die Rolle der Adjazenlotsen in der Ausbildung von Fluglotsen am Simulator erläutert.
Anschließend werden in Kapitel \ref{ch:rl_grundlagen} die theoretischen Grundlagen der Lösungsdomäne (\ac{RL}) und relevanten Algorithmen vorgestellt.
Im Kapitel \ref{ch:konzeption} werden die Definition von Zustands- und Aktionsräumen sowie der Belohnungsfunktion behandelt, sowie deren Zusammenhang mit verschienenen Konfliktlösungsmaßnahmen erläutert.
Kapitel \ref{ch:evaluation} widmet sich der Evaluation des \ac{RL}-Systems, in dem die durchgeführten Experimente und deren Ergebnisse präsentiert und analysiert werden.
Abschließend werden in Kapitel \ref{ch:zusammenfassung} die Ergebnisse der Arbeit zusammengefasst und ein Ausblick auf mögliche zukünftige Arbeiten gegeben.
