\chapter{Statistische Auswertung}
\label{ch:Statistische_Auswertung}

Die Evaluation von Reinforcement-Learning-Algorithmen stellt besondere Anforderungen an die statistische Auswertung. Aufgrund der inhärenten Stochastik des Trainingsprozesses – bedingt durch zufällige Initialisierungen und probabilistische Policies – besitzen einzelne Trainingsläufe nur begrenzte Aussagekraft. Um dennoch belastbare Schlussfolgerungen über die Leistungsfähigkeit der entwickelten Methoden ziehen zu können, führt dieses Kapitel robuste statistische Metriken ein. Die hier vorgestellte Methodik, insbesondere die Berechnung von Konfidenzintervallen mittels Bootstrap, bildet die analytische Basis für die in Kapitel \ref{ch:evaluation} durchgeführte Validierung der Ergebnisse.

\section{Konfidenzintervalle und Bootstrap nach \citet{statistical_rl_evaluation}}

Ein \ac{KI} beschreibt den Bereich, in dem der wahre (unbekannte) Wert einer Kennzahl mit einer vorgegebenen Wahrscheinlichkeit liegt. In dieser Arbeit verwenden wir 95\%-Konfidenzintervalle, um die Unsicherheit der geschätzten Performancemaße zu quantifizieren. Die Berechnung erfolgt vollständig empirisch mittels Percentile-Bootstrap, wie von \citet{statistical_rl_evaluation} für Reinforcement-Learning-Evaluationen empfohlen.

\paragraph{Bootstrap in zwei Sätzen.} Wir haben nur wenige Runs. Anstatt Verteilungsannahmen zu treffen, simuliert der Bootstrap viele virtuelle Datensätze durch Ziehen \emph{mit Zurücklegen} aus den vorhandenen Runs. Auf jedem virtuellen Datensatz wird dieselbe Kennzahl berechnet; die Streuung dieser Bootstrap-Kennzahlen liefert direkt das Konfidenzintervall.

\paragraph{Percentile-Bootstrap (Algorithmus).}
\begin{enumerate}
	\item Gegeben \(n\) unabhängige Runs einer Variante mit Kennzahlen \(p_1,\dots,p_n\).
	\item Ziehe \(B\) Stichproben der Größe \(n\) \emph{mit Zurücklegen} aus \(\{p_i\}\).
	\item Berechne für jede Stichprobe die Kennzahl (IQM oder Mittelwert der Success Rate).
	\item Das 95\%-KI sind die Perzentile (2.5, 50, 97.5) der entstehenden Bootstrap-Verteilung.
\end{enumerate}
Dieses Verfahren ist nicht-parametrisch, benötigt keine Normalverteilungsannahme und bleibt stabil bei schiefen Verteilungen und kleinen Stichproben.

\paragraph{Stratified Bootstrap nach \citet{statistical_rl_evaluation}.} Bei mehreren Tasks/Environments wird nicht global gemischt resampelt, sondern \emph{pro Task} separat gezogen und anschließend werden die Beiträge gleichgewichtet aggregiert. Dadurch dominiert keine „leichte“ Task das Gesamtergebnis. In dieser Arbeit erfolgt die Auswertung taskweise (z.B. nur \texttt{crossing\_planes\_multiHead} oder nur \texttt{LunarLander-v3}); daher ist kein zusätzliches Stratified-Resampling nötig. Würden mehrere Environments gemeinsam ausgewertet, wäre das stratifizierte Vorgehen erforderlich, um Verteilungsverschiebungen zu vermeiden.

\section{Robuste Lagekennzahl: Interquartile Mean (IQM)}

Statt einfacher Mittelwerte nutzen wir den Interquartile Mean (IQM) als robuste Lagekennzahl der episodischen Rewards. Der IQM mittelt nur über das mittlere 50\%-Quantil und dämpft damit Ausreißer deutlich \citep{statistical_rl_evaluation}. Für jede Variante wird der IQM der \emph{last\_eval\_mean}-Werte berechnet; das 95\%-Konfidenzintervall resultiert aus dem Percentile-Bootstrap mit \(B=5000\).

\section{Vergleich zweier Varianten}

Für zwei Algorithmen \(A\) und \(B\) werden getrennte Bootstrap-Verteilungen ihrer IQM-Werte erzeugt. Die Differenz-Distribution entsteht durch Ziehen \emph{beider} IQM-Werte pro Bootstrap-Iteration und anschließendes Bilden \(IQM_A - IQM_B\). Das daraus abgeleitete 95\%-Konfidenzintervall ermöglicht die Aussage, ob sich die Varianten signifikant unterscheiden. Liegt 0 innerhalb des Intervalls, ist kein signifikanter Unterschied nachweisbar; liegt das Intervall vollständig oberhalb (unterhalb) von 0, spricht dies für eine Überlegenheit von \(A\) (bzw. \(B\)).

\section{Success-Rate-Analyse}

Neben den Rewards wird die Erfolgsrate (Anteil erfolgreicher Episoden pro Evaluation) ausgewertet. Da es sich um bereits robuste Anteilswerte handelt, wird hier der arithmetische Mittelwert als Kennzahl verwendet. Ein 95\%-Konfidenzintervall des Mittelwerts wird wiederum per Percentile-Bootstrap (\(B=5000\)) bestimmt; der Unterschied zweier Varianten folgt analog durch Bootstrap der Mittelwertdifferenzen.

\section{Performance Profiles}

Zur ganzheitlichen Bewertung erzeugen wir Performance Profiles \citep{statistical_rl_evaluation}. Für ein Gitter von Schwellenwerten $\tau$ wird der Anteil der Runs berechnet, deren Score $> \tau$ liegt. Dies resultiert in einer Verteilungsfunktion, die den Anteil der erfolgreichen Läufe über verschiedene Schwellenwerte hinweg darstellt. Um die statistische Sicherheit dieser Kurve zu bewerten, werden mittels wiederholtem Resampling der Runs (Bootstrap) 95\%-Konfidenzbänder berechnet.. Performance Profiles zeigen auf einen Blick, wie viel Masse der Verteilung oberhalb bestimmter Qualitätsniveaus liegt und sind besonders aussagekräftig bei heterogenen oder breiten Verteilungen.
